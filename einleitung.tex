\chapter{Einleitung}

Die Anwendung \glqq Wortfinder\grqq{} ist ein Spiel, welches dem Spielprinzip von Wortopia nachempfunden ist. Im Gegensatz zu Wortopia ist diese Anwendung allerdings vollständig offline verfügbar.

%% Spiel prinzip
Das Spielprinzip ist sehr simpel. Es wird ein Raster mit Buchstaben angezeigt und in diesem müssen in einer vorgegebenen Zeit, möglichst viele Wörter gefunden werden. Zum finden der Wörter lassen sich die Buchstaben mit der Maus verbinden. Je nach Länge des Wortes und der initialen Spielzeit, gibt es für jedes Wort Punkte. Nach einer Partie kann der Spieler seinen Highscore speichern. Die Highscores werden dabei nur lokal gespeichert.


%% Funktionalität
\paragraph{Struktur und Funktionalität}
Zum einfacheren Verständnis werden kurz die Funktionalitäten der wichtigsten Klassen angerissen. Es handelt sich hierbei um eine WPF Anwendung. Die main Methode befindet sich in der Klasse \href{https://github.com/EinToni/Wortfinder/blob/main/Wortfinder/App.xaml.cs}{\textit{App}} und öffnet das Hauptfenster \href{https://github.com/EinToni/Wortfinder/blob/main/Wortfinder/MainWindow.xaml.cs}{\textit{MainWindow}}, in welchem gespielt wird. Die Klasse \href{https://github.com/EinToni/Wortfinder/blob/main/Wortfinder/GameManager.cs}{\textit{GameManager}} verwaltet das Spiel und delegiert Befehle vom \href{https://github.com/EinToni/Wortfinder/blob/main/Wortfinder/MainWindow.xaml.cs}{\textit{MainWindow}} an verschiedene weitere Klassen.


Einzelne Spiele werden im Vorhinein generiert und als \href{https://github.com/EinToni/Wortfinder/blob/main/Wortfinder/Game.cs}{\textit{Game}} Objekte gespeichert. Dadurch lässt sich anzeigen, wie viele Wörter maximal findbar sind. Diese Objekte beinhalten das jeweilige Buchstabenraster, wobei die Buchstaben nach ihrer Auftretenswahrscheinlichkeit im Deutschen zufällig generiert werden und alle findbaren Wörter (als \href{https://github.com/EinToni/Wortfinder/blob/main/Wortfinder/Word.cs}{\textit{Word}} Objekte) in diesem Raster. Als Referenz der gültigen Wörter wird eine heruntergeladene \href{https://github.com/EinToni/Wortfinder/blob/main/Wortfinder/wordListGerman.txt}{Liste} an deutschen Wörtern verwendet. Diese wurde nicht auf Richtigkeit oder Vollständigkeit überprüft, ist aber für die Funktionalität des Programmentwurfs ausreichend.
Die Buchstaben werden in der \href{https://github.com/EinToni/Wortfinder/blob/main/Wortfinder/LetterGenerator.cs}{\textit{LetterGenerator}} Klasse und die findbaren Wörter in der \href{https://github.com/EinToni/Wortfinder/blob/main/Wortfinder/WordGenerator.cs}{\textit{WordGenerator}} Klasse generiert. Verwaltet werden die einzelnen \href{https://github.com/EinToni/Wortfinder/blob/main/Wortfinder/Game.cs}{\textit{Game}} Objekte in der \href{https://github.com/EinToni/Wortfinder/blob/main/Wortfinder/GameLibrary.cs}{\textit{GameLibrary}} Klasse. 
Wird ein Spiel gestartet, wird das geladene \href{https://github.com/EinToni/Wortfinder/blob/main/Wortfinder/Game.cs}{\textit{Game}} Objekt entfernt und ein neues generiert. Wird die Anwendung kurz nach dem Start des Spiels geschlossen, läuft der separate Thread, welcher das neue Spiel generiert, gewollt weiter bis es fertig generiert wurde.


Für den Highscore zählt der \href{https://github.com/EinToni/Wortfinder/blob/main/Wortfinder/GameScore.cs}{\textit{GameScore}} die aktuellen Punkte, welche vom \href{https://github.com/EinToni/Wortfinder/blob/main/Wortfinder/GameScoreCalculator.cs}{\textit{GameScoreCalculator}} berechnet werden. Soll der Highscore gespeichert werden, wird ein \href{https://github.com/EinToni/Wortfinder/blob/main/Wortfinder/Score.cs}{\textit{Score}} Objekt erstellt und durch den \href{https://github.com/EinToni/Wortfinder/blob/main/Wortfinder/ScoreManager.cs}{\textit{ScoreManager}} verwaltet, sowie lokal verschlüsselt auf der Festplatte gespeichert.

\paragraph{Repositories}
Der Quellcode der Anwendung sowie die Unit Tests befinden sich in folgendem Repository:
\begin{itemize}
\item{\makebox[3.6cm]{\textbf{Quellcode:}\hfill}\href{https://github.com/EinToni/Wortfinder}{github.com/EinToni/Wortfinder}}
\end{itemize}

Nachfolgend sind noch die direkten Links zum Anwendungscode, zu den Tests und zur fertig kompilierten, ausführbaren Anwendung aufgelistet. Außerdem ist auch das Repository mit dieser Dokumentation verlinkt, damit unter Umständen manche Bilder im Original und vergrößert betrachtet werden können.

\begin{itemize}
\item{\makebox[3.6cm]{\textbf{Anwendungscode:}\hfill}\href{https://github.com/EinToni/Wortfinder/tree/main/Wortfinder}{github.com/EinToni/Wortfinder/tree/main/Wortfinder}}
\item{\makebox[3.6cm]{\textbf{Unit Tests:}\hfill}\href{https://github.com/EinToni/Wortfinder/tree/main/Wortfinder.XUnitTests}{github.com/EinToni/Wortfinder/tree/main/Wortfinder.XUnitTests}}
\item{\makebox[3.6cm]{\textbf{Release Version:}\hfill}\href{https://github.com/EinToni/Wortfinder/tree/main/Release}{github.com/EinToni/Wortfinder/tree/main/Release}}
\item{\makebox[3.6cm]{\textbf{Dokumentation:}\hfill}\href{https://github.com/EinToni/WortfinderDoku}{github.com/EinToni/WortfinderDoku}}
\end{itemize}

Für eine bessere Nachvollziehbarkeit, sind die meisten Verweise auf Klassen oder Commits in Repositories direkt als Link im Text hinterlegt.